%!TEX root = main.tex

%\secmoveup
\section{Estimating influence in retweet cascades}
\label{sec:user-influence}
%\secmoveup

%\verify{ \textbf{Comments from LX:}\\
%4/ 
%Sec 3.1 and 3.2 it seems better off to just call ``diffusion graph'' ``diffusion tree''. 
%this section should say that the tree respects temporal order,  it's unobserved / latent (and why we need to infer the tree)\\
%5/ 
%sec 3.2 path z, and $\phi$
%z should be defined as a sequence of nodes .. 
%Prob() is the probability of reaching $u_k$ from $u_i$, and $\phi$ would be the \textit{expected number of users reached} using a model of independent binomials to decide whether or not to take each hop in the path (instead of \textit{average number of reachable users} … unclear what it means)
%}

%\TODO{LX}{assuming the general motivation for inferring hidden tree + user influence was in intro. this sec first define cascade + tree + influence and proceed from there.}
%In Twitter, a {\em cascade} $C$ of size $n$ is defined as a series of messages $v_i$ sent by user $u_i$ at time $t_i$, i.e. $C=\{v_i(u_i, t_i)\}_{i=1}^n$.
%Here $v_1 = (u_1, t_1)$ is the initial message, and $v_1, \ldots, v_n$ with $t_1<\ldots<t_n$ are subsequent retweets or resposts of the initial message.

An {\em information cascade} $V$ of size $n$ is defined as a series of messages $v_i$ sent by user $u_i$ at time $t_i$, i.e. $V=\{v_i=(u_i, t_i)\}_{i=1:n}$.
Here $v_1 = (u_1, t_1)$ is the initial message, and $v_1, \ldots, v_n$ with $t_1<\ldots<t_n$ are subsequent reposts or relays of the initial message.
In the context of Twitter, the initial message is an original tweet and the subsequent messages are retweets of that original tweet (which by definition, are also tweets).
A latent retweet diffusion graph $G=(V,E)$ has the set of tweets as its vertexes $V$, and additional edges $E=\{(v_i, v_j)\}$ that represent that the $j^{th}$ tweet is a retweet of the $i^{th}$ tweet, and respects the temporal precedence $t_i<t_j$. 
Web data sources such as the Twitter API provide cascades, but not the
diffusion edges. 
Such missing data makes it challenging to measure a given user's contribution to the diffusion process.

%\TODO{LX}{now we need a precise definition of influence}
% now define influence

%\TODO{LX}{discussion: does $u$ feel more natural for denoting users in a graph than $v$ (vertex)?}


%\secmoveup
\subsection{Modeling latent diffusions}
\label{subsec:model-latent-diffusions}
%\secmoveup

\noindent\textbf{Diffusion scenarios.}
We focus on tree-structured diffusion graphs, i.e. each node $v_j$ has only one incoming link $(v_i, v_j)$, $i<j$. 
Denote the set of trees that are consistent with the temporal order in cascade $C$ as ${\cal G}$, we call each diffusion tree a \emph{diffusion scenario} $G \in {\cal G}$.
Fig.~\ref{fig:side:a} contains a cascade visualized as a star graph, attributing subsequent tweets to the first tweet at $t_1$.
Fig.~\ref{fig:side:b} shows four example diffusion scenarios consistent with this cascade.
The main challenge here is to estimate the influence of each user in the cascade, taking into account all possible diffusion trees.

\noindent\textbf{Probability of retweeting.}
%We define the probability of an edge $\mathds{P}(\{v_i, v_j\})$ as the likelihood that $u_j$ emitted tweet $v_j$ as a direct retweet of $v_i$.
For each tweet $v_j$, we model the probability of it being a direct descendant of each previous tweet in the same cascade as a weighted softmax function, defined by two main factors:
%% LX: the extra citations appear to be unecessary here
%as follows.
%In line with previous work~\cite{Mishra2016,Zhao2015,Shen2014}, we model two factors:
%\eat{in the likelihood of retweeting}:
firstly, users retweet \emph{fresh} content~\cite{Wu2007}.
We assume that the probability of retweeting decays exponentially with the time difference $t_j - t_i$;
secondly, users prefer to retweet locally influential users, known as preferential attachment~\cite{Barabasi2005,Rizoiu2017}.
We measure the local influence $m_i$ of a user $u_i$ using her number of followers~\cite{kwak2010twitter,Cha2010}.
%The second assumption is that user prefers retweet from a user who has a lot of followers. 
%The number of followers a user has often related to its number of retweet. 
%More followers, a user has more channels he can get to broadcast his tweet to others, which increasing the likelihood of seeing by other users~\cite{1,2}.
%So the probability of retweeting can be express in the following equation:
We quantify the probability that $v_j$ is a direct retweet of $v_i$ as:
\begin{equation} \label{eq:prob-edge-mt}
	p_{ij} = \frac{m_i e^{-r({t_j-t_i})}}{\sum_{k=1}^{j-1} m_k e^{-r({t_j-t_k})}}
\end{equation}
%This equation describe the probability that user $u_b$ retweet from user $u_a$ 
where 
%$t_a$ and $t_b$ represent the retweet time of $u_a$ and $u_b$ respectively;
$r$ is a hyper-parameter controlling the temporal decay. 
It is set to $r = 6.8 \times 10^{-4}$, tuned using linear search on a sample of 20 real retweet cascades (details in the supplement~\cite[annex~D]{supplemental}).
% and
%$m_i$ is the number of followers of user $u_i$ (the user of $v_i$).
%$e^{-r({t_{b}-t_{a}})}$ express the probability decrease in exponential.
%$G$ is one diffusion scenario which user $u_b$ is added to.
%The probability of a diffusion scenario is intimately linked to the probability of an edge.
%\TODO{LX}{this seems to be event rate not probability, i.e. not normalised}

%\secmoveup
\subsection{Tweet influence in a retweet cascade}
\label{subsec:user-influence-mt}
%\secmoveup

% \TODO{LX}{Is this a fair summary of the algorithm:\\ 
% * diffusion scenarios are the set of trees that are consistent with the cascade (order in time); \\
% * the algorithms marginalizes over all scenarios to estimate user influence; \\
% * the algorithm is efficient at $O(N^{XX})$ time; \\
% * it is non-obvious and potentially new because \ldots}

% assumptions + difficulties
We additionally assume retweets follow {\em independent conditional diffusions} within a cascade. 
%s
This is to say that conditioned on an existing partial cascade of $j-1$ retweets denoted as $V^{(j-1)}=\{v_k\}_{k=1}^{j-1}$ whose underlying diffusion scenario is $G^{(j-1)}$, the $j^{th}$ retweet is attributed to any of the $k=1,\ldots,j-1$ prior tweets according to Eq.~\ref{eq:prob-edge-mt}, and is independent of the diffusion scenario $G^{(j-1)}$. 
%s
For example, the $5^{th}$ tweet in the cascade will incur four valid diffusion trees for each of the diffusion scenarios for 4 tweets -- this is illustrated in Fig.~\ref{fig:add-one-edge}. 
%s
This simplifying assumption is reasonable, as it indicates that each user $j$ makes up his/her own mind about whom to retweet, and that the history of retweets is available to user $j$ (as is true in the current user interface of Twitter).
%s
It is easy to see that under this model, the total number of valid diffusion trees for a 5-tweet cascade is $1\cdot 2\cdot 3\cdot 4=24$, and that for a cascade with $n$ tweets is $(n-1)!$.

The goal for influence estimation for each cascade is to compute the contribution $\varphi(v_i)$ of each tweet $v_i$ {\em averaging} over all independent conditional diffusion trees consistent with cascade $V$ and with edge probabilities prescribed by Eq.~\ref{eq:prob-edge-mt}. 
%s
Enumerating all valid trees and averaging is clearly computationally intractable, but the illustration in Fig.~\ref{fig:add-one-edge} lends itself to a recursive algorithm. 

\textbf{Tractable tweet influence computation}
We introduce the \emph{pair-wise influence score} $m_{ij}$ which measures the influence of $v_i$ over $v_j$.
$v_i$ can influence $v_j$ both directly when $v_j$ is a retweet of $v_i$, and indirectly when a path exists from $v_i$ to $v_j$ in the underlying diffusion scenario.
Let $v_k$ be a tweet on the path from $v_i$ to $v_j$ ($i < k < j$) so that $v_j$ is a direct retweet of $v_k$.
$m_{ik}$ can be computed at the $k^{th}$ recursion step and it measures the influence of $v_i$ over $v_k$ over all possible paths starting with $v_i$ and ending with $v_k$.
Given the above independent diffusions assumption, the $m_{ij}$ can be computed using $m_{ik}$ to which we add the edge $(v_k, v_j)$.
User $u_j$ can chose to retweet any of the previous tweets with probability $p_{kj}, k < j$, therefore we further weight the contribution through $v_k$ using $p_{ij}$.
We consider that a tweet has a unit influence over itself ($m_{ii} = 1$).
Finally, we obtain that:
\begin{equation} \label{eq:Mij-mt}
m_{ij} =
\left\{
\begin{array}{ll}
	\sum^{j-1}_{k=i} m_{ik}p_{kj}^2 &,i < j \\
	1 & ,i = j \\
	0 & ,i > j.
\end{array}
\right.
\end{equation}

Naturally, $\varphi(v_i)$ the total influence of node $v_i$ is the sum of $m_{ij}, j > i$ the pair-wise influence score of $v_i$ over all subsequent nodes $v_j$. 
The recursive algorithm has three steps. 
\begin{enumerate}
	\item {\bf Initialization.} $m_{ij}=0$ for $i, j=1,\ldots,n, j \neq i$, and $m_{ii} = 1$ for $i=1,\ldots,n$;
	\item {\bf Recursion.} For $j=2, \ldots, n$;
	\begin{enumerate}
		\item For $k=1, \ldots, j-1$, compute $p_{kj}$ using Eq.~\eqref{eq:prob-edge-mt};
		\item For $i=1, \ldots, j-1$, $m_{ij} = \sum_{k=i}^{j-1} m_{ik}p_{kj}^2$;
	\end{enumerate}
	\item {\bf Termination.} Output $\varphi(v_i)= \sum_{k=i+1}^n m_{ik}$, for $i=1,\ldots,n$. 
\end{enumerate}

We exemplify this algorithm on a 3-tweet toy example. 
%s
%\oldversion{\hl{TO BE UPDATED}
Consider the cascade $\{v_1, v_2, v_3\}$.
When the first tweet $v_1$ arrives, we have $m_{11} = 1$ by definition (see Eq.~\eqref{eq:Mij-mt}).
%
% and denote $m_{ij}$ as a vector of the m-score values for each node for the first tweet (i.e., $v_1$). 
%s
%This is a all-zero vector by definition, as there are no retweets yet. 
%s
After the arrival of the second tweet, which must be retweeting the first, we have 
$m_{12}= m_{11} p^2_{12}=1$, and $m_{22} = 1$ by definition.
%to account for the influence of $v_1$, 
%and $\phi^{(2)}_2=\phi^{(2)}_3=0$. 
%s 
The third tweet can be a retweet of the first or the second, therefore we obtain: 
\begin{align*}
	m_{13} =& m_{11} p^2_{13} + m_{12} p^2_{23} \; ;\\
	m_{23} =& m_{22} p^2_{23} \; ;\\
	m_{33} =& 1 \;.
\end{align*}
The second term of $m_{13}$ accounts for the indirect influence of $v_1$ over $v_3$ through $v_2$.
%$\phi^{(3)}_1$ has three terms, the first term denotes the influence node $v_1$ already had on $v_2$, the second term denotes the direct influence of $v_1$ to $v_3$ had node 3 is a retweet of node 1. 
%The influence of $v_3$ itself is still zero: $\phi^{(3)}_3=0$. Moreover, 
This is the final step for a 3-node cascade. 
%The algorithm outputs $[\phi_i]_{i=1:3}=[\phi^{(3)}_i]_{i=1:3}$. 
%}

%As a sanity check, we can see that $\sum_{i=1}^n \varphi(v_i) = \sum_{j=1}^n \sum_{i=1}^{j-1} \Pd ((v_i,v_j)) = n-1$. In other words, the total influence across all nodes in a cascade of size $n$ is equal to $n-1$, or the number of retweets in this cascade. 
%In addition, the recursion above permits an in-place implementation, i.e., we incrementally update $\varphi(v_i)$ in step 2, and do not need to keep the intermediate influence values $\phi^{(k)}(v_i)$.
The computational complexity of this algorithm is $O(n^3)$.
There are $n$ recursion steps, and calculating $p_{ij}$ at sub-step (a) needs $O(n)$ units of computation, and sub-step (b) takes $O(n^2)$ steps.
In real cascades containing 1000 tweets, the above algorithm finishes in 34 seconds on a PC.
For more details and examples, see the online supplement~\cite[annex~B]{supplemental}.

%\oldversion{
%\textbf{Tweet influence over one diffusion scenario.}
%Let $z(v_i, v_k) \in G$ be a path in the diffusion scenario $G$ -- i.e a sequence of nodes which starts with $v_i$ and ends with $v_k$.
%$\mathds{P}(z(v_i, v_k))$ is the probability of reaching $v_k$ from $v_i$.
%We define $\varphi(v_i | G)$ the influence of $v_i$ in the scenario $G$ as the expected number of users reached from $v_i$ using a model of independent binomials to decide whether or not to take each hop in each path that starts with $v_i$.
%Formally:
%\begin{align}
%	\varphi(v_i|G) 	&= \mathds{E} \left[ \sum_{v_k \in V(G)} \mathds{1} \left\lbrace z(v_i,v_k|G) \right\rbrace \right] \nonumber \\
%					&= \sum_{v_k \in V(G)} \mathds{E}\bigg[\mathds{1} \left\lbrace z(v_i,v_k|G) \right\rbrace \bigg] \nonumber \\
%					&= \sum_{v_k \in V(G)} \mathds{P}(z(v_i,v_k|G)) \nonumber \\
%					&= \sum_{v_k \in V(G)} \prod_{(v_a, v_b) \in z(v_i, v_k)} \mathds{P}\big((v_a, v_b)\big) \label{eq:infl-in-a-scenario-mt}
%\end{align}
%where $\mathds{1} \left\lbrace z(v_i,v_k|G) \right\rbrace$ is a function that takes the value 1 when the path from $v_i$ to $v_k$ exists in $G$, and 0 otherwise.
%
%\textbf{Tweet influence over a retweet cascade.}
%
%
%We define $\varphi(v_i)$ the influence of $v_i$ over all valid diffusion scenarios $\cal G$:
%\begin{align}
%	\varphi(v_i) 	&= \sum_{G\in{\cal G}} \mathds{P}(G) \varphi(v_i|G) \nonumber \\
%				&= \sum_{G\in{\cal G}} \mathds{P}(G) \sum_{v_k \in V(G)} \mathds{P}\big( z(v_i,v_k|G) \big) \label{eq:brute-user-infl-mt}
%\end{align}
%Under the assumption that retweeting events (i.e. edges) occur independently one from another, the probability of a diffusion scenario is:
%\begin{equation} \label{eq:prob-scenario-mt}
%	\mathds{P}(G) = \prod_{(v_a, v_b) \in E} \mathds{P}\big((v_a, v_b)\big)
%\end{equation}
%
%It is intractable to directly evaluate Eq.~\eqref{eq:brute-user-infl-mt} (together with Eq.~\eqref{eq:infl-in-a-scenario-mt} and~\eqref{eq:prob-scenario-mt}), particularly due to the factorial number of diffusion scenarios in $\cal G$ (as shown in the online supplement~\cite[annex~A]{supplemental}).
%%\ref{si-sec:infl-derivation}
%For example, there are $10^{156}$ diffusion scenarios for a cascade of 100 retweets.
%We develop, in the next section, an efficient \eat{linear time} algorithm to compute the influence of all tweets\eat{in a retweet cascade}.
%
%%\verify{RJA comment: This might be basic question but is the probability threshold 0.5? i.e. estimated prob needs to be greater than 0.5 for there to be edge from from user i to j in retweet cascade?}
%%% MAR: not really. We don't make a decision based on probabilities, but we work in probability space.
%%RJA: OK I think I get it now.
%
%%\secmoveup
%\subsection{Tractable tweet influence computation}
%\label{subsec:casin-algo}
%%\secmoveup
%
%The key observation for a tractable computation of Eq.~\eqref{eq:brute-user-infl-mt} is that tweets $v_k$ are added sequentially at time $t_k$, to each diffusion scenario constructed at time $t_{k-1}$.
%Adding $v_k$ to a diffusion scenario $G$ has two effects.
%First, $k-1$ new diffusion scenarios are generated, as $v_k$ can attach to any of the existing $k-1$ nodes.
%Second, $v_k$ contributes only once to the tweet influence of each tweet that is found on the branch it attaches to and it does not make any other contributions at times $t > t_k$.
%This process is exemplified in Fig.~\ref{fig:add-one-edge}.
%Node $v_5$ is added to a given diffusion scenario, generating 4 new diffusion scenarios at time $t_5$.
%The nodes colored in red see their influence increase as a result of adding $v_5$.
%%We color in red the nodes whose influence increases as a result of adding node $v_5$.
%This allows to compute tweet influence incrementally, by updating $\varphi(v_i), i < k$ at each time $t_k$.
%We denote by $\varphi^k(v_i)$ the value of tweet influence of $v_i$ after adding node $v_k$.
%Thus, we only keep track of how tweet influence increases as nodes attach and we do not need to construct all diffusion scenarios.
%
%We define $M_{ik}$ as the contribution of $v_k$ to the tweet influence of $v_i$, over all possible diffusion scenarios.
%It can be shown that the influence of $v_i$ increases at time $t_k$ by $M_{ik}$:
%\begin{equation} \label{eq:iter-computation}
%	\varphi^k(v_i) = \varphi^{k-1}(v_i) + M_{ik},
%\end{equation}
%\begin{equation} \label{eq:Mij-mt}
%\text{ where } M_{ik} =
%\left\{
%\begin{array}{ll}
%	\sum^{k-1}_{j=1}M_{ij}\mathds{P}^2 \big( (v_j, v_k) \big) &,i < k \\
%	1 & ,i = k \\
%	0 & ,i > k.
%\end{array}
%\right.
%\end{equation} 
%See the online supplement~\cite[annex~B]{supplemental} for the complete derivation from Eq.~\eqref{eq:brute-user-infl-mt} to Eq.~\eqref{eq:iter-computation} and~\eqref{eq:Mij-mt}. %\ref{si-sec:efficient-algo}
%
%%\verify{RJA comment: Absolutely not suggesting any change to the above [no time anyway!] but I'm just wondering why a user is allowed to influence herself? I thought it would be $M_{ik} =0, i \geq k$. Doesn't have any impact on the comparative results, of course}
%%% MAR: we need it for the math. For comparative results, we plot percentiles anyway, so +1 simply puts everyone higher, but does not change order.
%
%\textbf{Recursive computation of tweet influence.}
%$M_{ik}$ can be alternatively interpreted as the influence of tweet $v_i$ over $v_k$.
%Consequently, we can compute $\varphi^k(v_i)$ the total influence of $v_i$ as the sum of 
%the individual influences of $v_i$ over each of the other nodes in the diffusion.
%% influence that $v_i$ has over all the other nodes in the cascade.
%This can be recursively computed as:
%\begin{align*}
%	\varphi^k(v_i) &= \varphi^{k-1}(v_i) + M_{ik} \nonumber \\
%				   &= \varphi^{k-2}(v_i) + M_{i(k-1)} + M_{ik} = \ldots = \sum_{j=1}^k M_{ij}. 
%\end{align*}
%
%\begin{algorithm}[tbp]
%\color{gray}
%  \caption{Compute influence matrix $M$, \hl{LX: $\Pd_{ij}^2$ and matrix $M$ are unclear}}
%  \label{alg1}
%  \begin{algorithmic}
%  \REQUIRE A set of $n$ retweets $\{ v_i = (m_i, t_i) | i = 1,\dots,n\}$
%  \REQUIRE Parameter $r$ -- temporal decay.
%  \ENSURE influence matrix $M$
%
%  \STATE Initialize matrix $P = [ P_{ij} = 0, \forall i, j = 1,\dots,n]$
%  \STATE Initialize matrix $M = [ M_{ij} = 0, \forall i, j = 1,\dots,n]$
%  \STATE $M_{11} = 1$
%
%  \FOR{$i=1$ \TO $n$}
%  		\FOR{$j=i$ \TO $n$}
%  			\STATE $P_{ij} = m_i e^{-r({t_j-t_i})}$
%  		\ENDFOR
%  \ENDFOR
%  %\STATE \hl{column normalize $P_{ij}$}
%  \STATE \rvx{$P_{ij} \leftarrow P_{ij}/(\sum_i P_{ij})$}
%  \STATE \hl{$P_{ij} = P_{ij} \cdot P_{ij}$ (element-wise multiplication)}
%  
%  \FOR{$j=2$ \TO $n$} {
%		\STATE \hl{$M_{[1..j-1, j]} = M_{[1..j-1, 1..j-1]} \times P_{[1:j-1,j]}$}
%		\STATE $M_{jj} = 1$	
%  }\ENDFOR
%
%  \end{algorithmic}
%  \label{alg:casin}
%\end{algorithm}
%
%\textbf{Computational complexity for estimating influence.}
%The scalable influence computation algorithm shown in Algorithm~\ref{alg:casin} uses two matrices:
%matrix $P = [ P_{ij} ]$, with $P_{ij} = \mathds{P}^2 \big( (v_i, v_j) \big)$ and matrix $M = [ M_{ij} ]$.
%It computes each column of $M$ sequentially, using Eq.~\eqref{eq:Mij-mt}.
%%Assuming that matrix multiplications are executed in $O(1)$,
%%The computation of matrix $M$ finishes after $n$ steps (or rows??). 
%Recursively computing each column in $M$ takes $O(n^2)$ multiplications/additions, the total computations complexity is $O(n^3)$. 
%In real cascades containing 1000 tweets, the above algorithm finishes in 34 seconds on a PC.
%%, where $n$ is the total number of retweets in the retweet cascade.
%For more details and examples, see the online supplement~\cite[annex~B]{supplemental}.
%%\ref{si-sec:efficient-algo}
%\eat{
%%%LX: moved the following info to Sec 3.1, as it doesn't describe "Computational complexity"
%Throughout the experiments in Sec.~\ref{sec:evaluation-influence}
% and~\ref{sec:results-findings}, we set the temporal decay hyper-parameter defined in Eq~\eqref{eq:prob-edge-mt} to $r = 6.8 \times 10^{-4}$, tuned using linear search on a sample of 20 real retweet diffusions (details in the online supplement~\cite[annex~D]{supplemental}). %\ref{si-sec:choose-temp-decay}
% }
%
%}

\subsection{Computing influence of a user}
\label{subsec:user-influence-measure}

Given $\mathcal{T}(u)$ -- the set of tweets authored by user $u$ --, we define the user influence of $u$ as the mean tweet influence of tweets $v \in \mathcal{T}(u)$:
\begin{equation} \label{eq:user-infl-casin}
	\varphi(u) = \frac{\sum_{v \in \mathcal{T}(u)} \varphi(v)}{|\mathcal{T}(u)|}, \mathcal{T}(u) = \{ v | u_v = u\}
\end{equation}
To account for the skewed distribution of user influence, we mostly use the normalization --  percentiles with a value of 1 for the most influential user our dataset and 0 for the least influential -- denoted $\varphi(u) \%$ .
%To account for the skewed distribution of user influence, the results presented in Sec.~\ref{sec:evaluation-influence} and~\ref{sec:results-findings} use the percentile normalized values of user influence with a value of 1 for the most influential user our dataset and 0 for the least influential.