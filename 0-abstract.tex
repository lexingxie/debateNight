%!TEX root = main.tex
%
Serious concerns have been raised about the role of `socialbots' in manipulating public opinion and influencing the outcome of elections by retweeting partisan content to increase its reach.
%RJA (26/2) - removed "arbitrarily" because suggests behaviour is random/non-strategic, but in fact there are political goals
%Socialbots, loosely defined as automated programs that control user profiles on social media, 
%are estimated to account for between 9-15\% of all Twitter profiles. 
%For a socialbot, getting other users to retweet their content increases the chances that this content will diffuse through the network and therefore shape and influence public discussion.
%are suspected of influencing public discussions by arbitrarily retweeting partisan content to increase its reach.
Here we analyze the role and influence of socialbots on Twitter by determining how they contribute to retweet diffusions.
We collect a large dataset of tweets during the 1st U.S. presidential debate in 2016 and we analyze its 1.5 million users from three perspectives: user influence, political behavior (partisanship and engagement) and botness.
First, we define a measure of user influence based on the user's active contributions to information diffusions, i.e. their tweets and retweets. 
%a function of the contributions to retweet diffusions.
Given that Twitter does not expose the retweet structure -- it associates all retweets with the original tweet --
we model the latent diffusion structure using only tweet time and user features, and we implement a scalable novel approach to estimate influence over all possible unfoldings.
%Given that the real retweet structure is latent, with only 
%, and implement a novel approach to estimate influence where the retweet cascade structure is latent and only tweet time and the user features are observable.
%We use the BotOrNot API to obtains the botness -- likelihood of being a bot -- for 1.5 million user accounts. into human and non-human and 
Next, we use partisan hashtag analysis to quantify user political polarization and engagement.
Finally, we use the BotOrNot API to measure user \emph{botness} (the likelihood of being a bot).
We build a two-dimensional ``polarization map'' that allows for a nuanced analysis of the interplay between botness, partisanship and influence.
We find that not only are socialbots more active on Twitter -- starting more retweet cascades and retweeting more -- but they are 2.5 times more influential than humans, and more politically engaged. 
%Based on differential usage of partisan hashtags, 
Moreover, pro-Republican bots are both more influential and more politically engaged than their pro-Democrat counterparts.
However we caution against blanket statements that software designed to appear human dominates politics-related activity on Twitter.
%and promoting pro-Republican messages dominated the Twitter information landscape during the U.S. presidential election. 
Firstly, it is known that accounts controlled by teams of humans (e.g. organizational accounts) are often identified as bots.
%we note that automated approaches to identifying socialbots may classify as bots accounts controlled by teams of humans (e.g. organizational accounts) and individual Twitter users who use automated scheduling. 
Secondly, we find that many highly influential Twitter users are in fact pro-Democrat and that most pro-Republican users are mid-influential and likely to be human (low botness).
%densest area of Republican users 