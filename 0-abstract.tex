%!TEX root = main.tex
%
Estimating user influence from retweet cascades is challenging
due to two reasons:
the first is data availability -- the diffusion structure is typically unobserved,
the second is data size -- the estimation algorithm needs to scale well to
many thousands or millions of users producing retweets.
This technical report revisits the mathematical model behind a recently proposed
influence estimation algorithm~\citep{Rizoiu2018a}.
%%LX - the sentence below contains no new info
%which operated in the absence of information concerning the diffusion structure.
In particular, we show that the particular choice of latent retweet probability
is motivated by self-exciting point processes.
We present an iterative algorithm that computes the probability of each
latent diffusion tree, and efficiently computes the expected number of retweets
of each tweet over all possibile diffusion trees.
Finally, we compare the empirical influence score distribution produced by this algorithm with the version in~\citep{Rizoiu2018a}.
%We define the pair-wise influence as the probability of a tweet being retweeted, and we compute the probability of a diffusion scenario -- a diffusion tree consistent with an observed discussion cascade.
%Finally, we compute the tweet influence as the expected number of retweets it obtains, over all possible diffusion trees.
%We detail a cubic complexity influence estimation algorithm, which scales to the amounts of data observed in social media originating data.
