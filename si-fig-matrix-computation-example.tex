%!TEX root = main.tex

\begin{figure*}[tbp]
	\centering
	\begin{minipage}{\textwidth}
		 \[
 \underbrace{\left[
\begin{matrix}
 \textcolor{gray}{1}  &\textcolor{blue}{m_{12}}  &\textcolor{red}{m_{13}}   &\textcolor{ForestGreen}{m_{14}}      & \cdots    & m_{1n} \\
 0                    &\textcolor{blue}{1}   	 &\textcolor{red}{m_{23}}   &\textcolor{ForestGreen}{m_{24}}      & \cdots    & m_{2n} \\
 0                    &0             			 &\textcolor{red}{1}   	    &\textcolor{ForestGreen}{m_{34}}      & \cdots    & m_{3n} \\
 0                    &0             			 &0             			&\textcolor{ForestGreen}{1}   		& \cdots    & m_{4n} \\
 \vdots    		      &\vdots            		 & \vdots       			&\vdots       				    & \ddots    & \vdots \\
 0         		      &0             			 &0                         &0             				    & \cdots    & 1      \\
\end{matrix}   
\right]}_{\textbf{M}} 
 \underbrace{\left[
\begin{matrix}
 1         &\textcolor{gray}{p_{12}}       &\textcolor{blue}{p_{13}}       &\textcolor{red}{p_{14}}        & \cdots    & p_{1n} \\
 0         &1             	    		   &\textcolor{blue}{p_{23}}       &\textcolor{red}{p_{24}}        & \cdots    & p_{2n} \\
 0         &0             				   &1					   		   &\textcolor{red}{p_{34}}         & \cdots    & p_{3n} \\
 0         &0             				   &0             				   &1					  		    & \cdots    & p_{4n} \\
 \vdots    &\vdots            			   & \vdots       				   & \vdots       				    & \ddots    & \vdots \\
 0         &0             				   &0                              &0             					& \cdots    & 1      \\
\end{matrix}   
\right]}_{\textbf{P}}
\]
\[
\begin{alignedat}{3}
&\begin{array}{rl}&m_{11} = 1 \end{array} &k =1\ \ \ \ \ &\\ 
&\begin{array}{rl}&m_{12} =p_{12}m_{11}\end{array} &k= 2 \ \ \ \ &\left[\textcolor{gray}{p_{12}}\right]\left[\textcolor{gray}{m_{11}}\right] = 
\left[\textcolor{blue}{m_{12}}\right]\\
&\begin{array}{rl}
	&m_{13} =p_{13}m_{11}+ p_{23}m_{12}\\
	&m_{23} =p_{13}m_{21}+ p_{23}m_{22}
\end{array}
\Bigg\} &k = 3 \ \ \ \ 
&\left[
\begin{matrix}
\textcolor{gray}{1}  &\textcolor{blue}{m_{12}} \\
0                    &\textcolor{blue}{1}   	
\end{matrix}
\right]
\left[
\begin{matrix}
\textcolor{blue}{p_{13}}\\
\textcolor{blue}{p_{23}}
\end{matrix}
\right]=
\left[
\begin{matrix}
\textcolor{red}{m_{13}}\\
\textcolor{red}{m_{23}}
\end{matrix}
\right]\\
&\begin{array}{rl}
	&m_{14} =p_{14}m_{11}+ p_{24}m_{12}+ p_{34}m_{13}\\
	&m_{24} =p_{14}m_{21}+ p_{24}m_{22}+ p_{34}m_{23}\\
    &m_{34} =p_{14}m_{31}+ p_{24}m_{32}+ p_{34}m_{33}
\end{array}
\Bigg\} &k = 4 \ \ \ \ 
&\left[
\begin{matrix}
 \textcolor{gray}{1}  &\textcolor{blue}{m_{12}}  &\textcolor{red}{m_{13}}\\
 0                    &\textcolor{blue}{1}   	 &\textcolor{red}{m_{23}}\\
 0                    &0             			 &\textcolor{red}{1}   	 
\end{matrix}
\right]
\left[
\begin{matrix}
\textcolor{red}{p_{14}}\\
\textcolor{red}{p_{24}}\\
\textcolor{red}{p_{34}}
\end{matrix}
\right]=
\left[
\begin{matrix}
\textcolor{ForestGreen}{m_{14}}\\
\textcolor{ForestGreen}{m_{24}}\\
\textcolor{ForestGreen}{m_{34}}
\end{matrix}
\right]\\
\end{alignedat}
\]
	\end{minipage}
	\caption{
		Exemplification of the efficient computation of the first four columns of matrix $M$.
%		Algorithm~\ref{alg:casin} which describes in the previous section can also be written in  matrix processing, The element of the matrix $m_{ij}$ can be obtained by the following steps:	
%		According to the equation(7), 
		Each $k^{th}$ column vector of matrix $M$ is colored correspondingly with the column vector in $P$ used in the multiplication.
%		
%		
%		can be computed by sub-matrix with size of $k-1\times k-1$ in $m_{ji}$ multiplying $k$th column vector with same color in $p_{ji}$.
%		Therefore, given matrix $p_{ji}$, from $k = 1$, $M$ can be computed incrementally follow the process above.		
	}
	\label{fig:matrix-operations-example}
\end{figure*}